Usually people use one or the other term to represent both errors and exceptions but these terms, while both having
the same root of something unexpected occurred, have a difference that exceptions can be handled,
while error shouldn't be handled.

For example, if we try to open a file which does not exist we should get an FileNotFoundException exception.
Normal application will handle that exception to inform the user about missing file and continue working normally.

Errors on the other way prevent the program from running or even compiling.
One of the frequent errors is SyntaxError which happens when we don't follow language constructs such as using
undeclared variable or misspelling the keywords.
This type of error is also called compile time error as it is thrown when compiling the application.
Other type of errors are runtime errors which happen during the application run.
Usually these errors would lead to app ending or restarting.
For example if we do not have enough memory on the system we will get OutOfMemoryError, and best the app can do
is to end/restart.

\subsection{REST Api Errors/Exceptions and HTTP Status Codes}
\label{subsec:rest-api-errors/exceptions-and-http-status-codes}

HTTP Status Codes are defined in \href{https://httpwg.org/specs/rfc9110.html#overview.of.status.codes}{RFC 9110}
which groups the statuses codes (100-599) in five categories:
\begin{itemize}
    \item 100--199 are informational codes (101 Switching Protocols)
    \item 200--299 are success codes (200 OK)
    \item 300--399 are redirect codes (301 Moved Permanently)
    \item 400--499 are client error code (404 Not Found)
    \item 500--599 are server error code (500 Internal Server Error)
\end{itemize}

REST APIs are able to return the result of the action or error/exception and most frameworks like NestJS are
tying the errors with server codes.
One example would be if we search the database by id and don't find the object we can throw NestJS NotFoundException
which will use 404 http status code.

\subsection{Recommendations for HTTP call errors}\label{subsec:recommendations-for-http-call-errors}

My recommendation is to use only two HTTP Status Codes:
\begin{itemize}
    \item 400 for all exceptions where client retries would not change the outcome.
    This should represent that user has sent wrong data or not allowed to do something.
    If something doesn't change this request would never succeed.
    \item 500 for all errors where client retries could succeed after some timeout.
    This might be thrown when cache is refreshing, or there is loss of network connectivity
    to some other services.
\end{itemize}

When exception is being returned to the HTTP client, in addition to HTTP Status Code,
we also need to return more data about the error in response body.
These response bodies should be more or less detailed depending on the environment zone the app is running.
In \bi{prod zone} we should output only exception id and generic name of the error,
while in \bi{dev zone} we can additionally send more data or stack trace.

Also, JavaScript and NestJS error implementation provide short description of the exception/error,
with \bi{message} property - which is not recommended.
Having many exceptions/errors with unique name is better for responding and providing proper error description with
the translation tooling.
Having errors translated on the client (frontend) is much preferable as the clients/frontends are translated
more often than backends.
We have frequently seen multilingual content on websites but rarely (if ever) translated versions of endpoint URLs or
POST data names.
Additional point is that if we have multiple clients (web and mobile app) each of those can define their own user
presentation of that error which is not possible with fixed description returning from backend.

Fine-grained exception/errors are preferred and naming should reflect those, and build from something larger
to something smaller.
For example \bi{AuthorizationTokenBearerMissingException}, \bi{AuthorizationTokenExpiredException}, \ldots
If possible we should group certain exceptions into one by providing additional parameters as we have entity field
in \bi{EntityNotFoundException(\{entity: User, id: '1234'\})}
and \bi{EntityNotFoundException(\{entity: UserProfile, id: '1234'\})}.

Exceptions/errors should be easy to make, and they should be grouped together with relevant
errors instead of all in one place or each for it self.

\subsection{Zeddy Errors}\label{subsec:zeddy-errors}

Zeddy Errors is npm package \href{https://www.npmjs.com/package/zeddy-errors}{https://www.npmjs.com/package/zeddy-errors}
that follows recommendations from \hyperref[sec:REST Api Errors/Exceptions and HTTP Status Codes]{REST Api Errors/Exceptions and HTTP Status Codes}.
It features most simple creation of new errors, allows groups of errors, ease of import, type safety and ease of use.
It has 0 dependencies except TypeScript and integrates well into any existing frameworks.
There is a sample NestJS project available at
\href{https://github.com/zveljkovic/zeddy-errors-example}(https://github.com/zveljkovic/zeddy-errors-example).

Usage is simple, export error group with defined errors with optional generic argument about the data type,
import it where needed and throw the new instance.

\begin{minted}{javascript}
// file: shared.errors.ts
export const SharedErrors = {
  EntityNotFound: class extends ExceptionBase<{ type: string; id: number }> {},
  JwtTokenMissingBearer: class extends ExceptionBase{},
  IntentionalError: class extends ErrorBase<{ reason: string }> {},
};

// file: some.service.ts
import {SharedErrors} from './shared.errors';
//...
throw new SharedErrors.IntentionalError({reason: "Testing errors"});
throw new SharedErrors.JwtTokenMissingBearer();

\end{minted}