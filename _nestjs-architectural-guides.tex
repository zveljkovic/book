Depending on the expected size and complexity, application architecture should be designed to maximise productivity,
and minimise complexity but without sacrificing the maintainability or development of possible future requirements.
There are various online topics on subject of architecture and their applicability, so we will shortly focus on most
common types of errors during architecture design.

Over-engineering: this happens when solution is unnecessarily complicated.
There are many examples of this but most common ones are:
\begin{itemize}
  \item designing for multiple databases in agnostic way when 99\% of projects never switch the database type (i.e.\ going from Postgresql to Sql Server).
  \item using microservice architecture where hidden cost of included complexity on both developer and devops side is not
  properly compared to the benefits of this approach.
  \item designing for infinite customisation of the components.
\end{itemize}

Under-engineering (Hacky): this happens when solution is not using parts for intended purposes but is trying to repurpose
existing ones in different way (which often fails).
Some of these are:
\begin{itemize}
  \item using database instead of in memory cache.
  \item using self-made code for distributed task scheduling (please use ZooKeeper).
  \item using cron jobs instead of proper background task processing.
\end{itemize}


This section will focus on most commonly used architectural practices for building NestJS based backend applications
in increasing complexity.

\section{As Simple As It Gets}\label{sec:arch-as-simple-as-it-gets}

The most basic setup we can have is to have one module that has one controller defined.
All the code gets written in the controller method.
This is only suitable for quick POC maybe when testing new libraries and later thrown away.
But even then it is a questionable practice as it might be easier to test it in main project.

\section{One module project based with services based on db entities}\label{sec:arch-om-db}

One module project will have no modularization but can have multiple controllers, services and other NestJS components.
Modularization is emulated with the folder structure and this works really well for POC or small projects.
Mainly for smaller projects there is 1:1 correlation between controllers and services.
Downside of this approach is that availability of each app component may tempt usage of classes that should be private
if there was proper modularization.
One example would be to update user directly via UserRepository instead of going through the UserService as it requires
all attributes to be specified.
Later, for example if we wanted to send a confirmation email everytime user is updated we would have to find all places
where UserRepository was used and add code there.

\section{Multi module project with services based on db entities}\label{sec:arch-mm-db}

An expansion on previous architecture is that now we are using multiple modules properly
(by forbidding imports of classes that are not exported from modules), but we still have groupings around entities.
This way we have multiple modules, well-defined public interfaces but if application grows in complexity we may start
encountering cyclic dependencies.
Cyclic dependencies mostly happen in services, and reason is that two service require each other to do the job.
For simplistic example we can say that UserService and WorkService have a cyclic dependency
when WorkService needs UserService to fetch the user, while UserService in deleteUser method needs WorkService
to remove all the Work entities for current user.

\section{Multi module project with services based on Domain Driven Design}\label{sec:arch-mm-ddd}

To solve the previous issue we can try the DDD and model services around business domains, and that might help a bit,
but when domain interactions become more complex we might still experience the cyclic dependency issue.

\section{Multi module project with services based on Domain Driven Design plus events}\label{sec:arch-mm-ddd-events}

To fight cyclic dependencies arising from large service classes, and make them independent of other modules we might
start to utilize internal events.
By raising events we may handle some cases in parts of other modules.
For example, in UserService we can use exported class UserEvents to raise UserCreated event, and then the
EmailService can respond to that event without needing a DI link to UserService.
This will help greatly in decoupling the modules and making app more maintainable for larger projects.

\begin{tcolorbox}[title=Dependency Injection issues]
  It becomes obvious at this point that dependencies should be scoped to the point of the usage.
  Having large service classes that depend on the other big service classes, we will come to the point where we
  will have to fight with cyclic dependencies.
  But it's not only that, why do we need to instantiate a service if it is used in only one method.
  We need a new (old) approach.
\end{tcolorbox}


\section{Multi module project with CQRS}\label{sec:arch-mm-cqrs}

There is another approach to fight large dependencies - CQRS\@.
CQRS stands for command query responsibility segregation and this pattern makes distinction (responsibility segregation)
between commands and queries.
Commands are used to mutate data (i.e.\ creates/updates) while queries are used where data is not updated
(i.e.\ retrieving objects).
This Nest.JS concept uses CommandBus/QueryBus pattern to run the commands/queries, but doesn't enforce any of the rules.
Queries executed this way can still mutate data so framework still expect the developer to follow the guidelines.

This pattern helps with scoping dependencies to single actions (command or query) as they are separate from the
rest of the actions even for the same resource.
Sample calling of the command and query is listed below, please see whole example project at
\href{htthttps://github.com/zveljkovic/book-cqrs-example}{https://github.com/zveljkovic/book-cqrs-example}.

\begin{minted}[xleftmargin=20pt,linenos]{javascript}
import { Body, Controller, Get, Post } from '@nestjs/common';
import { CommandBus, QueryBus } from '@nestjs/cqrs';
import { GetAllUsersQuery } from '../commands/get-all-users-query';
import { User } from '../entity/user';
import { CreateUserCommand } from '../commands/create-user-command';

@Controller('users')
export class UserController {
  constructor(private commandBus: CommandBus, private queryBus: QueryBus) {}

  @Get()
  async findAll() {
    return await this.queryBus.execute<GetAllUsersQuery, User[]>(
      new GetAllUsersQuery(),
    );
  }

  @Post()
  async create(@Body('id') id: number, @Body('name') name: string) {
    return await this.commandBus.execute<CreateUserCommand, User>(
      new CreateUserCommand(id, name),
    );
  }
}

\end{minted}

On line 3 and 5 we include empty POJO class, on line 9 we depend on the \bi{CommandBus} and \bi{QueryBus}
but not on the actual app dependencies.
This setup makes it possible that actions depend only on their own dependencies unlike services classes that
have dependencies for all of their methods.

Action handler for CreateUserCommand is displayed below to showcase the actual ICommandHandler interface.

\begin{minted}[xleftmargin=20pt,linenos]{javascript}
import { CommandHandler, EventBus, ICommandHandler } from '@nestjs/cqrs';
import { UserStore } from '../store/user-store';
import { User } from '../entity/user';
import { UserCreatedEvent } from '../events/user-created-event';
import { CreateUserCommand } from './create-user-command';

@CommandHandler(CreateUserCommand)
export class CreateUserCommandHandler
implements ICommandHandler<CreateUserCommand>
  {
  constructor(private userStore: UserStore, private eventBus: EventBus) {}

  async execute(command: CreateUserCommand) {
    const { id, name } = command;
    const user = new User();
    user.id = id;
    user.name = name;
    this.userStore.addUser(user);
    this.eventBus.publish(new UserCreatedEvent(user));
    return user;
  }
}
\end{minted}



