To better understand the available architectural guides in NestJS we need to shortly introduce ourselves with
core building blocks of NestJS\@.

\subsection{NestJS Modules}\label{subsec:nestjs-modules}

NestJS has is able to modularize the code with Module classes.
Each application must have at least one module but can have many modules.
These modules can contain multiple classes, and they have an interface with which we can specify
what is available from the modules importing it.
This allows us to have separation of concerns, by creating self-sufficient modules with public interfaces
(by specifying public classes in exports field of module).
For example, we can create UserModule that will provide UserService class to any module importing UserModule.
Parent modules will be able to get the UserService object via dependency injection system, but they couldn't
get the ``private'' classes used within module like UserRepository, UserEmailService, UserController and others unless
explicitly exported from the module.
This allows to hide internal implementation of the module and force usage of exported classes
to interact with this module.

\subsection{NestJS Providers}\label{subsec:nestjs-providers}
NestJS concept of provider is very large but can be sumarized as a class that can be injected through NestJS dependency
injection system.
Many of the NestJS basic classes are providers like services, repositories, helpers, controllers, etc\ldots
Class can be treated as a provider if it has @Injectable annotation and is defined in modules provider field.

\subsection{NestJS Controllers}\label{subsec:nestjs-controllers}

NestJS has built-in support for Controllers.
They are objects that define endpoints that allow the user to interact with the application.
For each endpoint controllers specify which URL, and HTTP method are required to trigger it.
Also, controllers serve as a place to define expected request parameters, validation, authorization, and most importantly
to trigger the specific code to handle that request.

\subsection{NestJS Middleware, Exception Filters, Pipes, Guards, Interceptors}\label{subsec:nestjs-others}
The rest of NestJS components are essentially a way to interact with request before it reaches controller.
Short explanation of each follow but for more in-depth explanation please read official documentation.

\begin{itemize}
  \item Middleware - code that runs before the routing.
  \item Exception filters - code that will allow you to react to unhandled exception from your code.
  \item Pipes - used for transformation and validation of data before execution of controller.
  \item Guards - checking if request is allowed to be executed by controller (authentication and authorization).
  \item Interceptors - wrap controller execution so you can write code that is executed before or after controller execution
\end{itemize}
